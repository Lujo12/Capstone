\chapter{Software Project Management Plans}


\section{Introduction}
 
\subsection{Project Overview}
 Our capstone, is part of a proposal to Defense Advanced Research Projects Agency (DARPA). This proposal wants to ''gives'' more intelligence to unmanned terrestrial vehicle. Our main focus is in the image segmentation. We receive images and classify them using a Probabilistic Neural Network (PNN) using 5 features: shape, color, location, normalized area and texture.\\
    
 The purpose of this document named Software Project Management Plan is to establish in a clear form, the details and specifications of how this Capstone would be administrated. This document is going to facilitate the organization of the project, and it would guarantee that all the requirements are being fulfilled for the completion of the product.  This documents discuss the following topics:
\begin{itemize}
  \item Evolution of the SPMP
  \item Project Organization
  \item Managerial Process
  \item Technical Process
  \item Work Packages, Schedule and Budget
\end{itemize}

The SPMP is directed to:

\begin{enumerate}
  \item Clients - DARPA
  \item Users - United States Armed Forces
  \item Developers - people responsible for maintenance and making updates to the system.
\end{enumerate}

\subsection{Project Deliverables}

\begin{itemize}
  \item IEEE 1058 Software Project Management Plan\\
            Due Date: \\
            Quantity: 1\\
            Localization: L-310 
  \item IEEE 830 Software Requirements Specification\\
            Due Date: \\
            Quantity: 1\\
            Localization: L-310 
  \item IEEE 829 Software Test Document \\
            Due Date: \\
            Quantity: 1\\
            Localization: L-310

  \item IEEE 1016 Software Design Description\\ 
            Due Date:\\
            Quantity: 1\\
            Localization: L-310 
  \item Code
\end{itemize}

\subsection{Evolution of the SPMP}

If any member of the group understands that the SPMP document needs to be updated for any reason, the member would have all the right to present a motion to express his point of view.  If the point of view is valid and is seconded by the two other members, the document would be updated.\\

Our main tool for updating this document is Github. Github is a repository where any member of the group can make changes without completely overwriting each other updates. As a second option, in case we experience any problems with Github, we will use dropbox to update this document.


\subsection{Reference Material}

IEEE Documents:
\begin{itemize}
  \item IEEE Std 1058.1-1987  Software Project Management Plan\\
				Author: The Software Engineering Technical Committee of the Computer Society of the IEEE\\
                Date: Approved December 10, 1987\\
				PDF: ISBN 0-7381-0409-4, SS12138

  \item IEEE Std 829-1998  Software Test Documentation\\
				Author: The Software Engineering Technical Committee of the Computer Society of the IEEE\\
                Date: Approved 16 September 1998
				PDF:ISBN 0-7381-1444-8 SS94687

  \item IEEE Std 830-1998 Software Requirement Specification \\
				Author: The Software Engineering Technical Committee of the Computer Society of the IEEE \\
                Date: Approved 25 June 1998\\
				PDF: ISBN 0-7381-0332-2

  \item IEEE Std 1016-1998 Software Design Description\\
				Author: The Software Engineering Technical Committee of the Computer Society of the IEEE\\
                Date: Approved 23 September 1998\\
				PDF: ISBN 0-7381-1456-1 SS94688	
\end{itemize}

Papers: 

\begin{itemize}
  \item Region-Based Image Retrieval Using Probabilistic Feature Relevance Learning\\
        Author: ByoungChul Ko, Jing Peng and Hyeran Byun
  \item Probabilistic Neural Networks SUpporting Multi-Class Relevance Feedback in Region-based Image Retrieval\\
        Author: ByoungChul Ko and Hyeran Byun\\
        Date: 2002
  \item FRIP: A Region-Based Image Retrieval Tool Using Automatic Image Segmentation and Stepwise Boolean AND Matching\\
        Author: ByoungChul Ko and Hyeran Byun\\
        Date: 2005
  \item A General Method for Unsupervised Segmentation of Images Using a Multiscale Approach\\
        Author: Alvin H. Kam and William J. Fitzgerald\\
        Date: 2000
\end{itemize}

Books:

\begin{itemize}
  \item Image Processing in C Second Edition\\
        Author: Dwayne Phillips
        Date: 2000
\end{itemize}

\subsection{Definitions and Acronyms}
\begin{table}[H]
\begin{tabular}{|c|c|}
  \hline
  % after \\: \hline or \cline{col1-col2} \cline{col3-col4} ...
  Term & Definition \\
   \hline
  a & b \\
   \hline
  c & d \\
   \hline
  \hline
\end{tabular}
\caption{Definitions}
\end{table}

\begin{table}[H]
\begin{tabular}{|c|c|}
  \hline
  % after \\: \hline or \cline{col1-col2} \cline{col3-col4} ...
  Term & Acronym \\
   \hline
  Defense Advanced Research Projects Agency  & DARPA \\
   \hline
  Software Requirements Specification & SRS \\
   \hline
    Software Project Management Plan & SPMP \\
   \hline
     Software Test Document & STD \\
   \hline
     Software Design Description & SDD \\
   \hline
\end{tabular}
\caption{Acronyms}
\end{table}
%\begin{table}[h]
%\centering
%\begin{tabular}{c c c}
%\hline
% & Pedro & Emanuel  \\ [0.5ex]
%\hline
%RAM Speed & 5559 MB/s & 2416 MB/s \\
%Floating Point Operations/Seconds & 102413776 & 25270774 \\
%Integer Operations/Seconds & 259364396
% & 56112514
% \\
%MD5 Hashes Generated/Second & 721235
% & 659176 \\
%3D Frames Per Second & 119 & 37 \\
%Primary Partition Capacity & 281 GB & 221 GB \\
%Drive Write Speed & 61 MB/s & 43 MB/s \\
%NovaBench Score & 502 & 251 \\
%\hline
%\end{tabular}
%\label{table:nonlin}
%\caption {NovaBench Results}
%\end{table}\\


%\begin{figure}[h]\centering
%  \includegraphics[width=6.0in]{Cores.jpg}\\
%  \caption{Testing how many cores use Microsoft Windows 7 for the lab program}\label{Cores}


%\begin{figure}[h]\centering
%  \includegraphics[width=\textwidth]{threads.jpg}\\
%  \caption{Three states of the threads}\label{threads}
%\end{figure}
